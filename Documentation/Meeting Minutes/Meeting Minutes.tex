\documentclass[]{article}

%opening
\title{ENGR489 Meeting Minutes \\ QuickCheck for Whiley}
\author{Janice Chin \\ Primary Supervisor: David Pearce \\ Secondary Supervisor: Lindsay Groves}
\date{}
\begin{document}
	
\maketitle

\section{March 20th 2018}
\subsection{Present}
\begin{itemize}
	\item David Pearce
\end{itemize}
\subsection{Action points were achieved since the last meeting}
None as this is the first meeting.
\subsection{Action points have yet to be achieved since the last meeting}
None as this is the first meeting.
\subsection{Action points were agreed to for the next meeting}
\begin{enumerate}
	\item Install Whiley command line tool.
	\item Read research papers about QuickCheck to determine strategies for generating test cases.
\end{enumerate}

\subsection{Discussion Points}

\paragraph{Objective 1:} Take Whiley program and generate test cases from it using the Whiley compiler. 

Assuming each Whiley program has specifications for each function.

To investigate:
\begin{itemize}
	\item Learn and understand Whiley
	\item Whiley jar file for the compiler
	\item Whiley web project
	\item Maven, pom.xml for dependencies for the generator
	\item Need to know strategies for generating test cases
\end{itemize}


\section{March 26th 2018}
\subsection{Present}
\begin{itemize}
	\item David Pearce
\end{itemize}

\subsection{Action points were achieved since the last meeting}

\begin{itemize}
	\item Read some research papers about QuickCheck and other testing tools (DART, ArbitCheck)
	\item Read Whiley Getting Started Guide and started reading language specifications
	\item Got WhileyLabs working on my machine, could not get Whiley working on the command line.
\end{itemize}
\subsection{Action points have yet to be achieved since the last meeting}
None
\subsection{Action points were agreed to for the next meeting}
\begin{itemize}
	\item Finish draft project proposal and send to supervisors to get it checked (by 30/3/2018).
	\item Reading more research papers. Suggest reading papers about American Fuzzy Lop.
	\item Begin implementation of project. The first steps are to create a Java project with the necessary hooks to access the Whiley compiler.  And, then to be able to read a compiled Whiley file (*.wyil) to identify functions to test. 
\end{itemize}

\subsection{Discussion Points}

Whiley can also pass functions as an argument to other functions.
\begin{itemize}
		\item What testing method am I using? Random testing? Property testing?  
		\begin{itemize}
			\item Firstly, use random testing using the properties of the function. Pre-condition to for candidate values and post-condition to check test passed or failed. Then will look at limitations of the testing method and decide whether to expand on the testing method (such as using dynamic symbolic execution).
		\end{itemize}
		\item What types to use in test generation? Primitive, recursive etc
		\begin{itemize}
			\item Primitive types: bool, byte, int, real, null, any, void
			\item Array
			\item Later on: records (closed), union
			\item Even later: Recursive types
			\item Extra (may not be implemented) - references, functionss
		\end{itemize}
		\item Testing functions only or also test methods? Methods have side effects.
		\begin{itemize}
			\item Test for functions first then by methods.
		\end{itemize}
		\item Using pre- and post-conditions or property syntax?
		\begin{itemize}
			\item Do not need to worry as the interpreter should evaluate the pre- and post-conditions. Properties are only used by the verifier and are specified in the pre- and post-conditions.
		\end{itemize}
\end{itemize}

\section{April 9th 2018}
\subsection{Present}
\begin{itemize}
	\item David Pearce
\end{itemize}

\subsection{Action points were achieved since the last meeting}
\begin{itemize}
	\item Read more research papers. Of particular note is MoreBugs, an extension to QuickCheck which generalises test cases so test cases that discover the same bug do not occur again.
	\item Created base implementation of QuickCheck for Whiley. This does
	\begin{itemize}
		\item Generates Int (specific Int) and random Bool
		\item Check candidate values are valid based on the precondition. Note: Does not generate a new test to replace the candidate values that do not pass the precondition.
		\item Validates tests using postcondition
	\end{itemize}
	\item Finished project proposal
\end{itemize}
\subsection{Action points have yet to be achieved since the last meeting}
None
\subsection{Action points were agreed to for the next meeting}
\begin{itemize}
	\item Start on writing background survey for preliminary report.
	\item Consider the design of QuickCheck for Whiley. Look into more iteration strategies for generating test values.
\end{itemize}

\subsection{Discussion Points}

\begin{itemize}

	\item Wanted to know about wyal.util.SmallWorldDomain as it also uses a Generator.
		\begin{itemize}
			\item Finds counterexamples for the Whiley Theorem Prover.
			\item Exhaustive iteration through possible values in a small range
		\end{itemize}
	\item Technique for writing a related works section in the report is to write a paragraph summary of each paper and then use those paragraphs as a basis for the section. Should be high level.
	\item Keep the test case generator flexible to allow different testing techniques.
	\item Want to compare random vs exhaustive testing.
	\item Function generation can be difficult. Can be inefficient if functions call other functions. Could just create return values for the functions that conform to the functions post-condition. However, need to consider tradeoff between performance and accuracy.
\end{itemize}

\section{April 16th 2018}
\subsection{Present}
\begin{itemize}
	\item David Pearce
\end{itemize}

\subsection{Action points were achieved since the last meeting}
	\begin{itemize}
		\item Implemented exhaustive test generation for booleans and integers.
		\item Started background survey but have not done much of this.
	\end{itemize}

\subsection{Action points have yet to be achieved since the last meeting}
Background survey for preliminary report
\subsection{Action points were agreed to for the next meeting}
\begin{itemize}
	\item Array, record and nominal generation
\end{itemize}

\subsection{Discussion Points}
\begin{itemize}
	\item Nominal types. Look at NameResolver and Flowtype checker. Ignore constraints when generating the tests. Like type synonyms.
	\item Bounded random state space. Selecting n samples, using basic probability across the state space.
	\item Arrays should be limited by its array size. This should be a constant upper limit. Random generation of elements. The arrays takes a generator element.
	\item Records. Print name of the field. Don't know field names of \texttt{...}. This generates n number of fields.
	\item Recursive types. Need to know if the type is recursive. Infinite data type e.g. binary tree. Limit it like an array. Cyclic generators? e.g. type list is \{int data, null | List next \}
	\item Integer range paper - For figuring out ranges to use during generation.
\end{itemize}

\section{April 30th 2018}
\subsection{Present}
\begin{itemize}
	\item David Pearce
\end{itemize}

\subsection{Action points were achieved since the last meeting}
\begin{itemize}
	\item Array generation. Size of elements are bounded between 0 and 3.
	\item Nominal type generation.
	\item Record generation for closed records only. 
	\item Null generation
\end{itemize}
\subsection{Action points have yet to be achieved since the last meeting}
None
\subsection{Action points were agreed to for the next meeting}
\begin{itemize}
	\item Union generator
	\item Start on integer range analysis
\end{itemize}
\subsection{Discussion Points}
\begin{itemize}
 \item ResolutionError occurs when a name of a type cannot be found. To resolve, need to import the correct files used such as the Whiley standard library.
 \item Look at Knuth's Algorithm S for random sampling.
 \item Null type primarily used for recursive types.
 \item When generating union types, need to fairly select values of each type. E.g. for int|bool, select an int then a bool then an int etc.
\end{itemize}

\section{May 7th 2018}
\subsection{Present}
\begin{itemize}
	\item David Pearce
\end{itemize}

\subsection{Action points were achieved since the last meeting}
\begin{itemize}
	\item Completed union generation
	\item Fixed some bugs with nominal invariants not being applied
	\item Completed integer range analysis ONLY for nominal types that wrap an integer. I.e. nat is (int x) where x \textgreater 0.  
\end{itemize}

\subsection{Action points have yet to be achieved since the last meeting}
None
\subsection{Action points were agreed to for the next meeting}
\begin{itemize}
	\item Run the valid and invalid tests written for testing the Whiley compiler.  To see what your tool does and to check that: valid tests don't produce counter examples, and invalid tests (ideally) do.  Don't expect every invalid test will find a counter example though.
	\item Continue working on integer range analysis
\end{itemize}
\subsection{Discussion Points}
\begin{itemize}
 \item Whiley does implicit casting. E.g. type nat is (int x) where 'a' \textgreater x will verify. In Whiley, a char is an integer from 0 to 255.
 \item A nominal type can have multiple invariants due to multiple where clauses.
 \item Implies (== \textgreater) is a 
 
 \item Each type could have a range. This means that the Generators for types would mirror the hierarchy for Ranges. An array would have a range for the elements itself and the length of the array. 
 \item Trying to add integer ranges for all elements in an array is difficult. For example, type intArr is (int[] x) where all \{ i in 0...\texttt{|x| |} x[i] \textgreater 0 \} .  Could possibly pattern match on this case?
 
 \item implies (== \textgreater) for x ==\textgreater y is !(x) \texttt{||} (x \&\& y)
 \item iff (\textless== \textgreater) for x \textless== \textgreater y is equality i.e. (x == y) 
\end{itemize}



%\section{ 2018}
%\subsection{Present}
%\begin{itemize}
%	\item David Pearce
%\end{itemize}
%
%\subsection{Action points were achieved since the last meeting}
%\subsection{Action points have yet to be achieved since the last meeting}
%\subsection{Action points were agreed to for the next meeting}
%
%\subsection{Discussion Points}
%\begin{itemize}
% \item Should integer range be able to do e.g. type weird is (int x) where x+1 > 0 . I have also limited my integer range to the form x ~ c
% \item Some of the Whiley valid tests are not detected, e.g. Coercion_Valid 1, 4,5,6,
% \item There are constraints as Whiley properties? (predicates in Dafny)
%\end{itemize}


\end{document}
