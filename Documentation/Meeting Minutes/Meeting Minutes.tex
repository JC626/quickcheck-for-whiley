\documentclass[]{article}

\usepackage{hyperref}
%opening
\title{ENGR489 Meeting Minutes \\ QuickCheck for Whiley}
\author{Janice Chin \\ Primary Supervisor: David Pearce \\ Secondary Supervisor: Lindsay Groves}
\date{}
\begin{document}
	
\maketitle

\section{March 20th 2018}
\subsection{Present}
\begin{itemize}
	\item David Pearce
\end{itemize}
\subsection{Action points were achieved since the last meeting}
None as this is the first meeting.
\subsection{Action points have yet to be achieved since the last meeting}
None as this is the first meeting.
\subsection{Action points were agreed to for the next meeting}
\begin{enumerate}
	\item Install Whiley command line tool.
	\item Read research papers about QuickCheck to determine strategies for generating test cases.
\end{enumerate}

\subsection{Discussion Points}

\paragraph{Objective 1:} Take Whiley program and generate test cases from it using the Whiley compiler. 

Assuming each Whiley program has specifications for each function.

To investigate:
\begin{itemize}
	\item Learn and understand Whiley
	\item Whiley jar file for the compiler
	\item Whiley web project
	\item Maven, pom.xml for dependencies for the generator
	\item Need to know strategies for generating test cases
\end{itemize}


\section{March 26th 2018}
\subsection{Present}
\begin{itemize}
	\item David Pearce
\end{itemize}

\subsection{Action points were achieved since the last meeting}

\begin{itemize}
	\item Read some research papers about QuickCheck and other testing tools (DART, ArbitCheck)
	\item Read Whiley Getting Started Guide and started reading language specifications
	\item Got WhileyLabs working on my machine, could not get Whiley working on the command line.
\end{itemize}
\subsection{Action points have yet to be achieved since the last meeting}
None
\subsection{Action points were agreed to for the next meeting}
\begin{itemize}
	\item Finish draft project proposal and send to supervisors to get it checked (by 30/3/2018).
	\item Reading more research papers. Suggest reading papers about American Fuzzy Lop.
	\item Begin implementation of project. The first steps are to create a Java project with the necessary hooks to access the Whiley compiler.  And, then to be able to read a compiled Whiley file (*.wyil) to identify functions to test. 
\end{itemize}

\subsection{Discussion Points}

Whiley can also pass functions as an argument to other functions.
\begin{itemize}
		\item What testing method am I using? Random testing? Property testing?  
		\begin{itemize}
			\item Firstly, use random testing using the properties of the function. Pre-condition to for candidate values and post-condition to check test passed or failed. Then will look at limitations of the testing method and decide whether to expand on the testing method (such as using dynamic symbolic execution).
		\end{itemize}
		\item What types to use in test generation? Primitive, recursive etc
		\begin{itemize}
			\item Primitive types: bool, byte, int, real, null, any, void
			\item Array
			\item Later on: records (closed), union
			\item Even later: Recursive types
			\item Extra (may not be implemented) - references, functions
		\end{itemize}
		\item Testing functions only or also test methods? Methods have side effects.
		\begin{itemize}
			\item Test for functions first then by methods.
		\end{itemize}
		\item Using pre- and post-conditions or property syntax?
		\begin{itemize}
			\item Do not need to worry as the interpreter should evaluate the pre- and post-conditions. Properties are only used by the verifier and are specified in the pre- and post-conditions.
		\end{itemize}
\end{itemize}

\section{April 9th 2018}
\subsection{Present}
\begin{itemize}
	\item David Pearce
\end{itemize}

\subsection{Action points were achieved since the last meeting}
\begin{itemize}
	\item Read more research papers. Of particular note is MoreBugs, an extension to QuickCheck which generalises test cases so test cases that discover the same bug do not occur again.
	\item Created base implementation of QuickCheck for Whiley. This does
	\begin{itemize}
		\item Generates Int (specific Int) and random Bool
		\item Check candidate values are valid based on the precondition. Note: Does not generate a new test to replace the candidate values that do not pass the precondition.
		\item Validates tests using postcondition
	\end{itemize}
	\item Finished project proposal
\end{itemize}
\subsection{Action points have yet to be achieved since the last meeting}
None
\subsection{Action points were agreed to for the next meeting}
\begin{itemize}
	\item Start on writing background survey for preliminary report.
	\item Consider the design of QuickCheck for Whiley. Look into more iteration strategies for generating test values.
\end{itemize}

\subsection{Discussion Points}

\begin{itemize}

	\item Wanted to know about wyal.util.SmallWorldDomain as it also uses a Generator.
		\begin{itemize}
			\item Finds counterexamples for the Whiley Theorem Prover.
			\item Exhaustive iteration through possible values in a small range
		\end{itemize}
	\item Technique for writing a related works section in the report is to write a paragraph summary of each paper and then use those paragraphs as a basis for the section. Should be high level.
	\item Keep the test case generator flexible to allow different testing techniques.
	\item Want to compare random vs exhaustive testing.
	\item Function generation can be difficult. Can be inefficient if functions call other functions. Could just create return values for the functions that conform to the functions post-condition. However, need to consider tradeoff between performance and accuracy.
\end{itemize}

\section{April 16th 2018}
\subsection{Present}
\begin{itemize}
	\item David Pearce
\end{itemize}

\subsection{Action points were achieved since the last meeting}
	\begin{itemize}
		\item Implemented exhaustive test generation for booleans and integers.
		\item Started background survey but have not done much of this.
	\end{itemize}

\subsection{Action points have yet to be achieved since the last meeting}
Background survey for preliminary report
\subsection{Action points were agreed to for the next meeting}
\begin{itemize}
	\item Array, record and nominal generation
\end{itemize}

\subsection{Discussion Points}
\begin{itemize}
	\item Nominal types. Look at NameResolver and Flowtype checker. Ignore constraints when generating the tests. Like type synonyms.
	\item Bounded random state space. Selecting n samples, using basic probability across the state space.
	\item Arrays should be limited by its array size. This should be a constant upper limit. Random generation of elements. The arrays takes a generator element.
	\item Records. Print name of the field. Don't know field names of \texttt{...}. This generates n number of fields.
	\item Recursive types. Need to know if the type is recursive. Infinite data type e.g. binary tree. Limit it like an array. Cyclic generators? e.g. type list is \{int data, null | List next \}
	\item Integer range paper - For figuring out ranges to use during generation.
\end{itemize}

\section{April 30th 2018}
\subsection{Present}
\begin{itemize}
	\item David Pearce
\end{itemize}

\subsection{Action points were achieved since the last meeting}
\begin{itemize}
	\item Array generation. Size of elements are bounded between 0 and 3.
	\item Nominal type generation.
	\item Record generation for closed records only. 
	\item Null generation
\end{itemize}
\subsection{Action points have yet to be achieved since the last meeting}
None
\subsection{Action points were agreed to for the next meeting}
\begin{itemize}
	\item Union generator
	\item Start on integer range analysis
\end{itemize}
\subsection{Discussion Points}
\begin{itemize}
 \item ResolutionError occurs when a name of a type cannot be found. To resolve, need to import the correct files used such as the Whiley standard library.
 \item Look at Knuth's Algorithm S for random sampling.
 \item Null type primarily used for recursive types.
 \item When generating union types, need to fairly select values of each type. E.g. for int|bool, select an int then a bool then an int etc.
\end{itemize}

\section{May 7th 2018}
\subsection{Present}
\begin{itemize}
	\item David Pearce
\end{itemize}

\subsection{Action points were achieved since the last meeting}
\begin{itemize}
	\item Completed union generation
	\item Fixed some bugs with nominal invariants not being applied
	\item Completed integer range analysis ONLY for nominal types that wrap an integer. I.e. nat is (int x) where x \textgreater 0.  
\end{itemize}

\subsection{Action points have yet to be achieved since the last meeting}
None
\subsection{Action points were agreed to for the next meeting}
\begin{itemize}
	\item Run the valid and invalid tests written for testing the Whiley compiler.  To see what your tool does and to check that: valid tests don't produce counter examples, and invalid tests (ideally) do.  Don't expect every invalid test will find a counter example though.
	\item Continue working on integer range analysis
\end{itemize}
\subsection{Discussion Points}
\begin{itemize}
 \item Whiley does implicit casting. E.g. type nat is (int x) where 'a' \textgreater x will verify. In Whiley, a char is an integer from 0 to 255.
 \item A nominal type can have multiple invariants due to multiple where clauses.
 \item Implies (== \textgreater) is a 
 
 \item Each type could have a range. This means that the Generators for types would mirror the hierarchy for Ranges. An array would have a range for the elements itself and the length of the array. 
 \item Trying to add integer ranges for all elements in an array is difficult. For example, type intArr is (int[] x) where all \{ i in 0...\texttt{|x| |} x[i] \textgreater 0 \} .  Could possibly pattern match on this case?
 
 \item implies (== \textgreater) for x ==\textgreater y is !(x) \texttt{||} (x \&\& y)
 \item iff (\textless== \textgreater) for x \textless== \textgreater y is equality i.e. (x == y) 
\end{itemize}

\section{May 13th 2018}
\subsection{Present}
\begin{itemize}
	\item David Pearce
\end{itemize}

\subsection{Action points were achieved since the last meeting}
\begin{itemize}
	\item Integer range analysis for integers in records and array sizes
	\item Tested QuickCheck on the Whiley Valid and Invalid tests with various statistics.
\end{itemize}

\subsection{Action points have yet to be achieved since the last meeting}
None
\subsection{Action points were agreed to for the next meeting}
\begin{itemize}
	\item Start introduction for preliminary report and a bit of the background survey.
	\item Integer range analysis for nominals
\end{itemize}

\subsection{Discussion Points}
\begin{itemize}
 \item Whiley properties(predicates in Dafny) are used for verification.
 \item Start writing preliminary report. You can look into expanding the project proposal.
 \begin{itemize}
 	 \item Introduction = The purpose of the project
 	\item Background = Briefly about Whiley? About other test frameworks.
 	\item Technical Discussion = High level. Could make a class diagram of the generators. Start with the basics first and then go into more depth such as the extensions of the tool (integer range analysis).
 	\item Data - What results you have so for
 	\item Request for feedback - Can ask if your method for evaluation is good? Evaluators may want more concrete evidence in the form of statistics.
 \end{itemize}
\end{itemize}


\section{May 21st 2018}
\subsection{Present}
\begin{itemize}
	\item David Pearce
\end{itemize}

\subsection{Action points were achieved since the last meeting}
\begin{itemize}
	\item Integer range analysis for nominals
	\item Started preliminary report - currently done 5 pages. 1 page introduction, 2 pages of background, 2 pages work completed.
\end{itemize}
\subsection{Action points have yet to be achieved since the last meeting}
None
\subsection{Action points were agreed to for the next meeting}
\begin{enumerate}
	\item Complete preliminary report
\end{enumerate}
\subsection{Discussion Points}
\begin{itemize}
	\item Introduction - Include what have you done so far? What you are doing? Could consider including objectives.
	\item Background 
	\begin{itemize}
		\item Should write about tools similar to QuickCheck. Consider looking into and writing about JCrasher.
		\item Write more about Whiley - relate how specifications can be used in testing. Static verification. Do not write about theorem prover but could specify about counter examples. Assume reader has never heard of and/or used Whiley before. 
	\end{itemize}
	\item Future plan
	\begin{enumerate}
		\item Evaluation of the tool needs to be done.
		\item Method/function calls within methods/functions are expensive. Therefore, need to optimise the performance of calling these functions/methods - call optimisation. One technique is instead of calling the function/method, is to just generate a return value. Problem if it is a method call as a method could modify a variable without returning it e.g. sorting an array without returning it.
		\item Performance comparision between executing functions/methods normally VS call optimisation approach.
		\item Mutation testing - mutating existing Whiley files. Mutating a file means to change some aspect of the file. Examples: changing operators, change forall to a sum, change if to a while statement, delete statements. This is to be able to check that the mutated file would have a different result than the original file. Where to apply mutation, on the .whiley or the .wyil file? Could do this on the .wyil file but then need to convert back into the .whiley file? 
		\item Could do recursive type generation, would need to limit the size of the value generated.
		\item Don't need to do function generation and open record generation. A possible future extension to the tool.
	\end{enumerate} 
	\item Sent draft of preliminary report to Dave, who has given feedback about it.
\end{itemize}

\section{May 28th 2018}
\subsection{Present}
\begin{itemize}
    \item David Pearce
\end{itemize}

\subsection{Action points were achieved since the last meeting}
Completed Preliminary Report
\subsection{Action points have yet to be achieved since the last meeting}
None
\subsection{Action points were agreed to for the next meeting}
\begin{itemize}
	\item Byte generator
	\item Use Algorithm S for random test generation
	\item Start looking into how functions/methods can be optimised
\end{itemize}
\subsection{Discussion Points}
\begin{itemize}
 \item No meeting for next week. Next meeting on June 11th. Same meeting time during exam period. 
 \item Fix typos, move listing captions below, add frames around code.
 \item Could expand on how generators work in QuickCheck
 \item Randoop more concentrated on constructors. Expand more about feedback-directed test generation. Think about talking how it generates values for parameters.
 \item Say more about random test generation, what was considered?
 \item Better explain generators
 \item Talk about issues encountered during implementation e.g. the difficulty of integer range analysis, recursive types, open records.
 \item Add time taken to run the tests. Why are two different integer ranges used in the tests (to limit number of combinations).
 \item Example test techniques in request for feedback
\end{itemize}

\section{June 11th 2018}
\subsection{Present}
\begin{itemize}
	\item David Pearce
\end{itemize}

\subsection{Action points were achieved since the last meeting}
\begin{itemize}
	\item Byte generator
	\item Implemented Algorithm S for random test generation. Was not sure if this should be added in as there could be bias in the results due to sampling without replacement.
	\item Bounded recursive type generation - bounded to a depth of 3
\end{itemize}
\subsection{Action points have yet to be achieved since the last meeting}
None
\subsection{Action points were agreed to for the next meeting}

\begin{itemize}
	\item Start optimising function calls. This is by replacing the execution of the function/method with a randomly generated value. If same function is called again with the same arguments, it should ideally return the same value.
\end{itemize}

\subsection{Discussion Points}
\begin{itemize}
	\item Algorithm S - random testing. Would be good to talk about the design and tradeoffs in the final report! Included should be fairness issue, performance cost, how it was implemented. 
	\item For random testing, it might be good to look at \url{http://homepages.ecs.vuw.ac.nz/%7Edjp/files/GPCE17_preprint.pdf}
	\item Record returned values for functions. Same inputs == same outputs. executeInvoke should be called. 
	Override the interpreter.
	\item How is it affecting the data for function optimisation? Invalid tests - are you finding the bugs?
	\item Ideally by the end of the mid-tri break (in tri2), should finish experiments for evaluation.
	\item Mutation testing should be done before the mid-tri break.
	\item QuickCheck versus verification for finding bugs and performance is interesting.
	\item Future extensions: References and methods
	\item This code does not verify, gets empty type error.
	Due to spaces VS indentation.
	
	\begin{verbatim}
	type Fun is function(int) -> int
	
	function map(int[] items, Fun fn) -> int[]
	requires |items| > 0 
	requires all { i in 0..|items| | items[i] > 0}:
	    int i = 0
	    while i < |items|
	    where 0 <= i && i <= |items|:
	        items[i] = fn(items[i])
	        i = i + 1
		return items
	\end{verbatim}
\end{itemize}


\section{June 18th 2018}
\subsection{Present}
\begin{itemize}
	\item David Pearce
\end{itemize}

\subsection{Action points were achieved since the last meeting}
 Created function optimisation for un-recursive functions by random generation.
\subsection{Action points have yet to be achieved since the last meeting}
None.
\subsection{Action points were agreed to for the next meeting}

\begin{itemize}
	\item Start preliminary work with the experiments. Record statistics when running tests, and compare function call optimisation and caching against Whiley test suite.
	\item Run Whiley tests to see if there are invalid tests. Also run WyBench tests from the develop branch.
	\item Add flags for different parameters like whether to enable and disable function optimisation.
\end{itemize}

\subsection{Discussion Points}
\begin{itemize}
 \item Encountered several problems when trying to implement function optimisation
 \begin{enumerate}
 	\item For the Interpreter, I had to copy a lot of methods into my own version of the Interpreter due to the private scope.
 	Ideally, the methods would have the protected scope so I would only need to override the methods I require.
 	Dave says this should be okay.
 	\item Recursive functions are a problem! Especially when the invariants call a recursive function as this causes a StackOverflow due to an infinite loop. Currently, this has been disabled if the function is calling itself (e.g. factorial function).
 	Not too sure how to solve this as I need to be able to detect the difference between a normal function call and recursive function call. 
 	E.g. if there are two functions foo() and bar() where foo calls bar and bar calls foo until some condition is met.
 	Should try and disable this.
 \end{enumerate}
\end{itemize}

\section{July 3rd 2018}
\subsection{Present}
\begin{itemize}
	\item David Pearce
\end{itemize}

\subsection{Action points were achieved since the last meeting}
\begin{itemize}
	\item Began experimenting on Whiley valid and invalid tests.
	\item Fix bugs discovered from testing.
	\item Attempted to test the WhileyBench tests, could not get this to work due to other libraries required during compiling. 
\end{itemize}
\subsection{Action points have yet to be achieved since the last meeting}

\subsection{Action points were agreed to for the next meeting}
\begin{itemize}
	\item Execute the WyBench tests. See if function optimisation impacts the performance of these tests.
	Require Dave to send the jar and wyil files for the remaining WyBench tests.
	\item Fix more bugs discovered from testing.
\end{itemize}
\subsection{Discussion Points}
\begin{itemize}
 \item Having problems importing libraries during compiling. Needed for the WyBench tests. 	Dave managed to get the wystd library working by using the correct version.
 \item How to format command line arguments as input? Not necessary to format command line arguments nicely. Only would be useful if it was widely used.
\end{itemize}

\section{July 16th 2018}
\subsection{Present}
\begin{itemize}
	\item David Pearce
\end{itemize}

\subsection{Action points were achieved since the last meeting}
\begin{itemize}
	\item Able to run the WyBench tests by using the Standard Library and the wybench.jar file provided.
	\item Discovered a few bugs in the WyBench tests.
\end{itemize}
\subsection{Action points have yet to be achieved since the last meeting}
None
\subsection{Action points were agreed to for the next meeting}
\begin{itemize}
	\item For the bugs discovered, add issues to the WyBench GitHub. Then fix these issues by submitting a pull request. This can then be addressed as evidence in the final report.
	\item Fix any further bugs in the tool, adding missing
	specifications and refactoring where appropriate.
	\item Investigate mutation testing. Setup the framework to read in a WyIL file, apply a mutation to some expression within that file, and the print out the file using
	wyc.io.WhileyPrettyPrinter. Copy the code for that from GitHub into your project and fix it.
\end{itemize}

\subsection{Discussion Points}
\begin{itemize}
	\item Received feedback on the Progress Report. Dave said not to worry too much about the feedback. Especially the function optimisation since I have done this already. 
	\item Presentation on Friday July 20th for the Programming Language user group at 12pm, CO255. Prepare a presentation between 5 to 10 minutes long, around 3 slides. Short introduction, maybe present data and explain about the generators.
	\item Take an Whiley as input, produce mutated file as output (Whiley). Consider how many to apply mutations to apply and where they should be applied. Could implement this as a applyMutation function, where you take an arbitrary expression and mutate it. Check the mutated file can be compiled.
\end{itemize}


\section{July 24th 2018}
\subsection{Present}
\begin{itemize}
	\item David Pearce
\end{itemize}

\subsection{Action points were achieved since the last meeting}
\begin{itemize}
	\item Added issues on GitHub to fix Whiley tests
	\item Fixed more bugs in the tool, and better error reporting when a test fails
	\item Started looking into mutation testing.
\end{itemize}
\subsection{Action points have yet to be achieved since the last meeting}
None
\subsection{Action points were agreed to for the next meeting}
\begin{itemize}
	\item Generating references - do simple cases first then consider what happens with aliasing.
	\item Generating functions - should be similar to function optimisation.
	\item Testing methods
\end{itemize}

\subsection{Discussion Points}
\begin{itemize}
	\item In the report, write about caching values (memoisation). How it affects performance?
	\item References should start at a random value (RValue.Cell). The Reference points to RValue.Cell.
	\item Problem with references is aliasing where two references can point to the same cell.
	For example, e.g. foo(\&int, \&int, \&int) can have different combinations such as 1, 1, 1 (all point to same cell), 0, 1, 0 (first and last point to same cell) etc.
	\item Have enough tests for the evaluation. Leave out mutation testing for now as it seems like another side project. If we do decide to do this, then we may do a quick implementation of it (e.g. just replacing operators).
\end{itemize}

\section{July 31st 2018}
\subsection{Present}
\begin{itemize}
	\item David Pearce
\end{itemize}

\subsection{Action points were achieved since the last meeting}
\begin{itemize}
	\item Can now test on methods
	\item Reference generation without aliasing
	\item Function generation - just generates a return value randomly using the return parameters. Similar to generating a record.
\end{itemize}
\subsection{Action points have yet to be achieved since the last meeting}
None
\subsection{Action points were agreed to for the next meeting}
\begin{itemize}
	\item Record statistics from method calls.
	\item Want to try and optimise performance of the tool, by reducing the number of tests if all combinations have been exhausted. Only for functions as the methods could alter global variables, therefore, the number of method calls matter.
	\item Add extra command line arguments for memoisation.
	\item Try and prepare structure for the final report.
\end{itemize}

\subsection{Discussion Points}
\begin{itemize}
	\item Next week, will start the final report. Could take parts of the progress report and put it in the final report
	\item Have a lot of material for evaluation, the statistics from testing and the issues raised on GitHub.
	\item Need to check if the return type for the function generated follows the type constraints. E.g. function(int) -> (nat) where the nat needs to be an int \textgreater= 0.
	\item For the functions generated, their return calls could also be memosised? (Hmmm... the function always returns the same value, so  calling a generated function shouldn't be costly?). Since this is not implemented, this can be added as a future extension in the report.
	\item Also in the report, can talk about why aliasing was not done for reference generation and how it could have been approached.
\end{itemize}

\section{August 7th 2018}
\subsection{Present}
\begin{itemize}
	\item David Pearce
\end{itemize}

\subsection{Action points were achieved since the last meeting}
\begin{itemize}
	\item Added flag for memoisation. Turns out it makes testing a lot faster. Due to methods being executed.
	\item Stop test execution if all possible test combinations have been generated.
	\item Allow memoisation on Properties too
	\item Started looking into the final report.
	\item Started executing statistics for methods. Still need to do some more due to some tests taking a long time. May need to tweak function optimisation.
\end{itemize}
\subsection{Action points have yet to be achieved since the last meeting}
None
\subsection{Action points were agreed to for the next meeting}

\begin{itemize}
	\item Start final report. Concentrate on the implementation and evaluation sections. Want to have a vision of what the final report will look like.
	\item Want to know how you interpret the various statistics in the evaluation. For example, could have the statistics in a table. GitHub issues could also be represented as a table of what issue was found, how it was identified, where it was corrected etc.
	\item Send draft to Dave of what was written in the morning. (Plan to send this through by Monday morning, Tuesday morning at the latest!)
\end{itemize}

\subsection{Discussion Points}
\begin{itemize}
	\item Realised I didn't use Knuth Algorithm S for random test case generation. So I implemented it. Problem: Very slow when there are a few tests but a large number of combinations such as when conducting function optimisation. As it has to iterate through X number of combinations, using probability to get the N number of tests. Previously, it would just iterate through the generators (where generators did execute Algorithm S), calling generate.
	\item Could output statistics into a latex format to be easily copied and pasted into the report.
	\item Bullet point what you want to write in the relevant sections in the report.
	\item Limitations part based on feedback on the progress report. Testing can't find everything however, verification can. What are the limitations of the tools in the research paper? Other limitations include the performance of the tool.
\end{itemize}


\section{August 14th 2018}
\subsection{Present}
\begin{itemize}
	\item David Pearce
\end{itemize}

\subsection{Action points were achieved since the last meeting}
\begin{itemize}
	\item Started implementation chapter - need to write function optimisation and memoisation
	\item Started figuring out what to do for the evaluation
	\item Attempted to fix random test problems - turns out that a high number of combinations 751,747,177 takes 12 seconds to execute 1 test. For 100 tests that is ~20 minutes long. Therefore, need another way to select tests, maybe randomly generate and add to a set until it reaches the number of tests required?
\end{itemize}
\subsection{Action points have yet to be achieved since the last meeting}
None
\subsection{Action points were agreed to for the next meeting}
\begin{itemize}
	\item Continue writing the report 
	\item Revise report based on feedback on the draft version of the report.
	\item Split implementation into design chapter. Design is a high level of what you have done. In the implementation can then why you did it in such and such way. For example, fairness of unions.
	\item Make diagrams stand out better
\end{itemize}
rename Integer range analysis
\subsection{Discussion Points}
\begin{itemize}
	\item Don't worry about the random test generation time problem. This can be talked about in the report.
	\item Investigate why memoisation is failing for the Whiley invalid tests.
\end{itemize}

\section{August 21st 2018}
\subsection{Present}
\begin{itemize}
	\item David Pearce
\end{itemize}

\subsection{Action points were achieved since the last meeting}
Wrote more of the report - especially the design and implementation chapters.
\subsection{Action points have yet to be achieved since the last meeting}
None
\subsection{Action points were agreed to for the next meeting}
Received feedback on report. Continue writing the report, the most feedback was on the introduction and background chapters so focus on that.
\subsection{Discussion Points}
\begin{itemize}
	\item  Should memoisation be applied to methods? As the number of method calls can affect the values in global variables. This is a design question that you can explain in your report, why you decided to do/not to do this.
	\item For statistics in the evaluation, you can either get the average from multiple tests or get the variable coefficient which is the standard deviation divided by the mean.
\end{itemize}


%\section{ 2018}
%\subsection{Present}
%\begin{itemize}
%	\item David Pearce
%\end{itemize}
%
%\subsection{Action points were achieved since the last meeting}
%\subsection{Action points have yet to be achieved since the last meeting}
%\subsection{Action points were agreed to for the next meeting}
%
%\subsection{Discussion Points}
%\begin{itemize}
%	\item Where to explain about how aliasing could be implemented? In the future work section?
%\end{itemize}


\end{document}
