\section{Future Plan}\label{section:future}

%This should highlight the main components which remain to be done, and provide a proposed time-line in which this will happen. In putting together a time line, students must take into account upcoming examinations, coursework deadlines
%and other disruptions.

A working implementation has been completed halfway through the project. However, there are still several components that need to be completed.
A timeline of these components is shown in Table \ref{table:timeline}

\begin{table}[H]
	  \centering
\begin{tabular}{ |p{10cm}|p{2cm}|p{2cm}|p{2cm}| }
\hline
\textbf{Output} & \textbf{Estimated Time} & \textbf{Start Date} & \textbf{Complete by}\\
\hline
Optimise functions/method calls when testing a specific function/method to improve the performance of the tool. & 3 weeks & 11/6/18 & 2/7/18 (Exam period) \\
\hline
Produce slides for presentation of preliminary report & 1 week & 2/7/18 & 9/7/18 (Break)\\
\hline
Add mutation testing by mutating Whiley files and comparing results from the mutated files with the original file. & 3 weeks & 9/7/18 & 30/7/18 (Week 3)\\
\hline
Produce draft of final report & 6 weeks & 30/7/18 & 15/9/18 (Week 7)\\
\hline
Evaluate the automatic test-case generator using the different test-case generation techniques against a test suite. & 3 weeks & 15/9/18 & 5/10/18 (Week 10) \\
\hline
Finalise final report & 2 weeks & 6/10/18 & 21/10/18 (Week 12) \\
\hline
Prepare slides for final presentation & 3 weeks & 22/10/18 & 16/11/18 (Exam period)\\
\hline
\end{tabular}

\caption{Timeline}
\label{table:timeline}

\end{table}

% Generation - byte, reference, recursive types? 