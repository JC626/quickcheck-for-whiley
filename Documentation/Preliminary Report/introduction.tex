\chapter{Introduction}\label{chapter:intro}
%This should briefly outline the project and if necessary
%reevaluate the original plan in light of what has been learned in the interim. In particular, any significant deviations in the problem being addressed, or the solution
%being developed should be clearly highlighted and justified.

%The motivation for this project is to ...

% What is Whiley
% What is automatic test case generation

The detection and elimination of bugs is important in software development as it improves the quality and reliability of software.

To help eliminate errors in code, the Whiley programming language was developed to verify code using formal specifications.
Ideally, programs written in Whiley should be error free.
To achieve this goal, Whiley contains a verifying compiler which employs the use of specifications written by a developer to check for common errors such as accessing an index of an array which is outside its boundaries \cite{WhileyLang}.

Listing \ref{lst:whileyMin} illustrates an example of a Whiley program with a function,  \texttt{min()} which finds the minimum value of two integers. Executing \texttt{min(2, 3)} on the program will return the integer, 2.

\begin{lstlisting}[language=Whiley, tabsize=3, numbers=left,
label={lst:whileyMin}, caption={Whiley program for the min function}]
function min(int x, int y) -> (int r)
ensures r == x ==> x <= y
ensures r == y ==> y <= x:
if x <= y:
	return x
else:
	return y
\end{lstlisting}

Currently, the verifying compiler has limitations when evaluating complex pre- and post-conditions.
For example in Listing \ref{lst:whileySquare}, the post-condition is falsely identified as not holding by the verifying compiler even though the program does meet the post-condition.

\begin{lstlisting}[language=Whiley, tabsize=3, numbers=left,
label={lst:whileySquare}, caption={Whiley program for the square function}]
function square(int x) -> (int r)
	ensures r >= 0:
	return x*x
\end{lstlisting}

Consequently, tests would need to be created to detect if there are further problems within the program. However, writing and running tests can be tedious and costly. Furthermore, it is difficult to write tests for all possible cases and be able to detect all bugs. Instead of manually creating tests, a tool could be created that automatically generates and executes tests on a program such as QuickCheck which was implemented in Haskell by Koen Claessen and John Hughes. \cite{QClightweight}


This project aims to implement an automated test-case generator for the Whiley programming language, based on the QuickCheck tool. As a result, the automatic test-case generator will improve software quality and increase confidence in unverifiable code. 


% TODO Talk about objectives within the project? Such as the Proposed Solution section of the project proposal
