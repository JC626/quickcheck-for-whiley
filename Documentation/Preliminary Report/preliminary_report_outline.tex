%% $RCSfile: proj_report_outline.tex,v $
%% $Revision: 1.3 $
%% $Date: 2016/06/10 03:41:54 $
%% $Author: kevin $

\documentclass[11pt
              , a4paper
              , oneside
              , openright
              ]{article}


\usepackage{float} % lets you have non-floating floats

\usepackage[usenames, dvipsnames]{color}
\usepackage{url} % for typesetting urls
\usepackage{pdfpages} % for importing the project proposal
\usepackage{listings}  % for code

\definecolor{light-gray}{gray}{0.93}

\lstdefinelanguage{Whiley}{
	keywords={assert, for, while, switch, is, if, case, return,
		else, process, define, as, requires, ensures, where, no,
		all, bool, int, byte, void, real, in, any, null, char,
		string, function, var, method, type, some, reduce, import
	},
	keywordstyle=\color{Plum},
	backgroundcolor = \color{light-gray},
	frame=single,
	basicstyle=\ttfamily,
	commentstyle=\rmfamily\itshape\color{gray},
	stringstyle=\small\itshape,
	moredelim=*[s][commentstyle]{/*}{*/}, % allows keyword highlighting inside comments
	morecomment=[l][commentstyle]{//},      % single line comments are set by...
	moredelim=*[s][\color{blue}]{"}{"}, % highlighting print statement
}

%
%  We don't want figures to float so we define
%
\newfloat{fig}{thp}{lof}[section]
\floatname{fig}{Figure}

%% These are standard LaTeX definitions for the document
%%                            
\title{QuickCheck for Whiley Preliminary Report}
\author{Janice Chin}

%% This file can be used for creating a wide range of reports
%%  across various Schools
%%
%% Set up some things, mostly for the front page, for your specific document
%
% Current options are:
% [ecs|msor|sms]          Which school you are in.
%                         (msor option retained for reproducing old data)
% [bschonscomp|mcompsci]  Which degree you are doing
%                          You can also specify any other degree by name
%                          (see below)
% [font|image]            Use a font or an image for the VUW logo
%                          The font option will only work on ECS systems
%
\usepackage[image,ecs]{vuwproject}

% You should specifiy your supervisor here with
     \supervisors{David Pearce and Lindsay Groves}
% use \supervisors if there is more than one supervisor

% Unless you've used the bschonscomp or mcompsci
%  options above use
\otherdegree{Bachelor of Engineering with Honours}
% here to specify degree

% Comment this out if you want the date printed.
\date{}

\begin{document}

% Make the page numbering roman, until after the contents, etc.
\frontmatter

%%%%%%%%%%%%%%%%%%%%%%%%%%%%%%%%%%%%%%%%%%%%%%%%%%%%%%%

%%%%%%%%%%%%%%%%%%%%%%%%%%%%%%%%%%%%%%%%%%%%%%%%%%%%%%%

\begin{abstract}
%A short description of the project goes here.
This report is about the progress accomplished in the project: QuickCheck for Whiley. 
The project aims to improve software quality of Whiley programs by implementing automated test-case generation. 
The Whiley language and the automatic test-case generation tools: QuickCheck, Concholic Unit Testing Engine and Randoop are investigated.
QuickCheck for Whiley has been implemented for random and exhaustive test-case generation with different data types generated and integer range analysis applied using constraints.
A timeline is created for completing the implementation of function/method call optimisation, mutation testing and evaluation of the tool. 
Feedback for evaluating the tool is requested.
\end{abstract}

%%%%%%%%%%%%%%%%%%%%%%%%%%%%%%%%%%%%%%%%%%%%%%%%%%%%%%%

\maketitle

\tableofcontents

% we want a list of the figures we defined
%\listof{fig}{Figures}

%%%%%%%%%%%%%%%%%%%%%%%%%%%%%%%%%%%%%%%%%%%%%%%%%%%%%%%

\mainmatter

%%%%%%%%%%%%%%%%%%%%%%%%%%%%%%%%%%%%%%%%%%%%%%%%%%%%%%%

% individual chapters included here
\section{Introduction}\label{section:intro}
%This should briefly outline the project and if necessary
%reevaluate the original plan in light of what has been learned in the interim. In particular, any significant deviations in the problem being addressed, or the solution
%being developed should be clearly highlighted and justified.

%The motivation for this project is to ...

% What is Whiley
% What is automatic test case generation

The detection and elimination of bugs is important in software development as it improves the quality and reliability of software.

To help eliminate errors in code, the Whiley programming language was developed to verify code using formal specifications \cite{WhileyLang}.
Ideally, programs written in Whiley should be error free \cite{WhileyLang}.
To achieve this goal, Whiley contains a verifying compiler which employs the use of specifications written by a developer to check for common errors such as accessing an index of an array which is outside its boundaries \cite{WhileyLang}.

Listing \ref{lst:whileyMin} illustrates an example of a Whiley program with a function,  \texttt{min()} which finds the minimum value of two integers. Executing \texttt{min(2, 3)} will return the integer, 2.

\begin{lstlisting}[language=Whiley, tabsize=3, numbers=left,
label={lst:whileyMin}, caption={Whiley program for the min function}]
function min(int x, int y) -> (int r)
ensures r == x ==> x <= y
ensures r == y ==> y <= x:
if x <= y:
	return x
else:
	return y
\end{lstlisting}

Currently, the verifying compiler has limitations when evaluating complex pre- and post-conditions.
For example in Listing \ref{lst:whileySquare}, the post-condition is falsely identified as not holding by the verifying compiler even though the program does meet the post-condition.

\begin{lstlisting}[language=Whiley, tabsize=3, numbers=left,
label={lst:whileySquare}, caption={Whiley program for the square function}]
function square(int x) -> (int r)
	ensures r >= 0:
	return x*x
\end{lstlisting}

Consequently, tests would need to be created to detect if there are further problems within the program. However, writing and running tests can be tedious and costly. Furthermore, it is difficult to write tests for all possible cases and be able to detect all bugs. Instead of manually creating tests, a tool could be created that automatically generates and executes tests on a program such as QuickCheck which was implemented in Haskell by Koen Claessen and John Hughes \cite{QClightweight}.

QuickCheck uses property-based testing and random testing \cite{QClightweight}. Properties defined by the user are used to randomly generate and execute an arbitrary number of tests \cite{QClightweight}. 

Random testing has been widely studied in the literature, leading to the creation of new variants and testing techniques including concolic testing \cite{CUTE}, feedback-directed testing \cite{randoopAll}, \cite{randoopJava} and mutation testing \cite{evoSuite}.

%Other tools have drawn inspiration from QuickCheck with variations in testing techniques including concolic testing \cite{CUTE}, feedback-directed testing \cite{randoopAll}, \cite{randoopJava} and mutation testing \cite{evoSuite}.

This project aims to implement an automated test-case generator for the Whiley programming language, based on the QuickCheck tool. As a result, the automatic test-case generator will improve software quality and increase confidence in unverifiable code. 

Currently, a working implementation of QuickCheck for Whiley has been completed which can generate and execute tests using random and exhaustive test-case generation. 
Constraints applied to integers and array sizes on nominal types have also been implemented to limit the range of integers and size of arrays generated.

% TODO Talk about objectives within the project? Such as the Proposed Solution section of the project proposal

\section{Background Survey}\label{section:background}

%This should discuss any existing solutions to the given problem, and may reference academic papers, books and other sources as appropriate. Care should be taken to identify key differences between these solutions, and that being developed in the project.

There exists a number of different test case generation tools for different languages such as QuickCheck for Haskell \cite{QClightweight}, Concolic Unit Testing Engine (CUTE) for C \cite{CUTE} and Randoop for Java \cite{randoopJava} and .NET \cite{randoopAll}.
QuickCheck for Whiley is specifically created for the Whiley programming language and uses different techniques to generate test data instead of just one technique.

\subsection{Whiley}
Whiley is a hybrid imperative and functional programming language developed by David Pearce \cite{WhileyPlatform}. The presentation of Whiley resembles an imperative language like Python whereas the core structure is functional and pure \cite{WhileyPlatform}. 

Functions in Whiley are pure therefore, an input will always result in the same output without any observable side effects \cite{WhileyLang}.
Methods are impure so may have side effects on input parameters or other aspects of the program \cite{WhileyLang}. 
Functions, methods and other compound data types are passed by value meaning they are a copy of the original variable when they are passed as an argument to a function or method \cite{WhileyPlatform}.


A key component in Whiley is the automatic verifying compiler
which is used in conjunction with explicit specifications to verify programs are correct \cite{WhileyPlatform}.
Specifications can be written as pre-conditions (\texttt{requires} clauses) and post-conditions (\texttt{ensures} clauses) in a function or method declaration \cite{WhileyPlatform}.
Specifications can also be written as invariants using \texttt{where} clauses on user defined types (nominal type) or as an invariant on a loop \cite{WhileyPlatform}. 

To be able to execute a Whiley file, it must first be compiled into a WyIL file which is the bytecode form of Whiley in the Whiley Intermediate Language \cite{WhileyPlatform}.
During compilation, the verifying compiler checks all specifications are met and checks for common errors such as division by zero \cite{WhileyPlatform}. Any errors or failed specifications are detected and reported to the developer to fix. Counter examples are also used within Whiley to notify how the specification can fail however, this functionality is limited to examples within a small range.
The WyIL file can then be executed if it verifies successfully or compiled without verification.

\subsection{QuickCheck}
QuickCheck is a tool implemented by Koen Claessen and John Hughes \cite{QClightweight} to test Haskell programs.
To check if a program is correct, QuickCheck uses property-based testing which uses conditions that will always hold true \cite{QClightweight}.
Users define the conditions as Haskell functions in terms of formal specifications to validate functions under test \cite{QClightweight}. 

QuickCheck then generates a large number of test cases automatically with random testing \cite{QClightweight}. 
Input values are randomly generated for each test by using generators that correspond to the input value's type \cite{QClightweight}. 
For quantitative types, a generator generates a value within a defined range such as between -5 and 5 for integers \cite{QClightweight}. 
For qualitative types, the generator uniformly selects a value that can be generated for that type such as generating true or false for a Boolean \cite{QClightweight}.
QuickCheck contains generators for most of Haskell's predefined types and for functions \cite{QClightweight}. 
User-defined types require the tester to specify their own custom generators with a bounded size if the type is recursive such as a binary tree. \cite{QClightweight}. 

A problem with random testing is the distribution of test data. Ideally, the distribution should adhere to the distribution of actual data which is often uniformly distributed \cite{QClightweight}. 
For example, executing tests for Listing \ref{lst:quickcheckExecute} could be skewed towards test data with short lists. 
The distribution of data is controlled by the tester by creating a custom generator for the data type with weighted frequencies on the methods used to generate the data \cite{QClightweight}.

An example of using QuickCheck would be reversing a list.
Firstly, a property is defined in Listing \ref{lst:quickcheckProperty} where the list \texttt{xs}, should be the same as reversing \texttt{xs} twice.

\begin{lstlisting}[language=haskell, label={lst:quickcheckProperty},
caption={Property for reversing a list in QuickCheck}, captionpos=b, frame = single]
propReverseTwice xs = reverse (reverse xs) == xs
\end{lstlisting}

QuickCheck is then executed by importing the property and passing it into the interpreter as shown in Listing \ref{lst:quickcheckExecute}.

\begin{lstlisting}[label={lst:quickcheckExecute}, caption={Executing tests to check a list is reversed}, captionpos=b,
frame = single]
Main:> quickCheck propReverseTwice
Ok: passed 100 tests.
\end{lstlisting}

A large number of test cases are created where random input values are generated to check the user-defined property holds. QuickCheck reports various test statistics including the number of tests that have passed or failed \cite{QClightweight}.

\subsection{Concolic Unit Testing Engine (CUTE)}
Concolic Unit Testing Engine (CUTE) generates test inputs using a combination of symbolic and concrete execution \cite{CUTE}. 
The aim of CUTE is to provide input values to explore all feasible execution paths of a program and thus achieve high path coverage \cite{CUTE}. CUTE has been applied to C programs \cite{CUTE}.

Random testing may generate inputs with the same behaviour leading to its redundancy \cite{CUTE}. Furthermore, the probability of selecting inputs that will detect errors using random testing is small \cite{CUTE}. 
Therefore, CUTE incrementally generates concrete inputs using symbolic execution to discover feasible paths \cite{CUTE}.

Firstly, CUTE executes the program with arbitrary inputs using concrete and symbolic execution concurrently \cite{CUTE}.
In concrete execution, the program is executed normally with the input values \cite{CUTE}.
At the same time, symbolic execution is run through the same path  \cite{CUTE}. 
Using the symbolic variables, constraints from branching expressions are discovered and stored \cite{CUTE}.
The last constraint applied is negated so that the next test run will explore a different feasible path \cite{CUTE}. 
The constraint is solved to limit and determine the possible inputs to execute in the next test \cite{CUTE}.

For example, we execute Listing \ref{lst:cuteExample} and create the random values \texttt{x = y = 1}. 
In concrete execution, \texttt{z} is set to 1 whereas in symbolic execution, \texttt{z} is set to \texttt{x * y}.
At line 2, \texttt{x != 2} so the condition fails. 
As the program went through the path where \texttt{x != 2}, a different path will be taken in the next test. This is by negating and solving the condition by setting \texttt{x} to 2 and y to 1. 
The test will then fail at line 4 as \texttt{x >= z}. Therefore, CUTE will then find values that solve the constraints \texttt{x == 2} and \texttt{x < z} such as \texttt{x = 2} and \texttt{y = 2}. 
Running this test will then exhibit the error on line 5. 


%Firstly, CUTE executes the program with arbitrary inputs using concrete and symbolic execution concurrently \cite{CUTE}.
%For example, we execute Listing \ref{lst:cuteExample} and create the random values \texttt{x = y = 1}. 
%In concrete execution, the program is executed normally with the input values \cite{CUTE}.
%At the same time, symbolic execution is run through the same path  \cite{CUTE}. 
%In concrete execution, \texttt{z} is set to 1 whereas in symbolic execution, \texttt{z} is set to \texttt{x * y}.
%Using the symbolic variables, constraints from branching expressions are discovered and stored \cite{CUTE}. At line 2 in the example, \texttt{x != 2} so the condition fails. 
%The last constraint applied is negated so that the next test run will explore a different feasible path \cite{CUTE}. 
%The constraint is solved to limit and determine the possible inputs to execute in the next test \cite{CUTE}.
%As the test failed the last condition as \texttt{x != 2}, the condition is negated and solved so in the next test, \texttt{x} is set 2 and y remains set to 1. The test will fail at line 4 as \texttt{x >=z}. Therefore, CUTE will then find values that solve the constraints \texttt{x == 2} and \texttt{x < z} such as \texttt{x = 2} and \texttt{ y = 2}. Running this test will then exhibit the error on line 5. 

\begin{lstlisting}[language=C, tabsize=3, numbers=left,
label={lst:cuteExample}, caption={Example C program}, captionpos=b,
frame=single]
void foo(int x, int y) {
	int z = x * y;
	if(x == 2){
		if(x < z){
			ERROR;
		}
	}
}
\end{lstlisting}

\subsection{Randoop}
Randoop which stands for random tester for object-oriented programs, generates unit tests using feedback-directed random test generation \cite{randoopAll}, \cite{randoopJava}. Method sequences from previous tests are used to help generate subsequent tests \cite{randoopAll}, \cite{randoopJava}.
As a result, a test suite is outputted with successful and unsuccessful tests \cite{randoopAll}, \cite{randoopJava}.

Randoop tests classes by executing a sequence of methods as a test \cite{randoopJava}. Method sequences are created incrementally by randomly selecting method calls and using arguments from previous sequences \cite{randoopJava}.
The sequence is then executed and checked against contracts \cite{randoopAll}.
Contracts are built into Randoop or optionally defined by the user \cite{randoopAll}.
If a sequence breaks a contract, then the test is outputted as a contract-violating test \cite{randoopAll}. If a sequence was successful, then the test is outputted as a regression test \cite{randoopAll}.
Successful sequences that are not redundant and can be extended are used in subsequent tests \cite{randoopAll}.


For example, executing Randoop on the Java program in Listing \ref{lst:randoopJavaProg} will produce several tests based on different method sequences.
One successful list of method sequences could be Listing \ref{lst:randoopSuccess},  which can be re-used in subsequent sequences. It is outputted into a regression test as shown in Listing \ref{lst:randoopSuccessTest}.
By extending the method sequences, another list of method sequences is created such as Listing \ref{lst:randoopError}.
This list of method sequences throws an error as the \texttt{equals()} method is incorrect as \texttt{a1} is not equal to \texttt{b1} and thus, is a contract-violating test that will be outputted.

\begin{lstlisting}[language=Java, tabsize=3, numbers=left,
label={lst:randoopJavaProg}, caption={Example Java class},
captionpos=b, frame=single]
public class A {
	public A(){ }
	@Override
	public boolean equals(Object obj){
		return true;
	}
}
public class B{
	public B(){ }
}
\end{lstlisting}

\begin{lstlisting}[language=Java, tabsize=3, numbers=left,
label={lst:randoopSuccess}, caption={Successful method sequence for testing Listing \ref{lst:randoopJavaProg}},
captionpos=b, frame=single]
A a1 = new A();
a1.equals(a1);
\end{lstlisting}

\begin{lstlisting}[language=Java, tabsize=3, numbers=left,
label={lst:randoopSuccessTest}, caption={Test output for the successful method sequence in Listing \ref{lst:randoopSuccess}},
captionpos=b, frame=single]
@Test
public void test1() {
	A a1 = new A();
	assertTrue(a1.equals(a1));
}
\end{lstlisting}

\begin{lstlisting}[language=Java, tabsize=3, numbers=left,
label={lst:randoopError}, caption={Contract violating method sequence for testing Listing \ref{lst:randoopJavaProg}},
captionpos=b, frame=single]
A a1 = new A();
a1.equals(a1);
B b1 = new B();
a1.equals(b1);
\end{lstlisting}

%A commercial version of QuickCheck in Erlang called Quviq QuickCheck is co-founded by one of the original developers, John Hughes and is an extension of the original QuickCheck for Haskell \cite{QCFunProfit}.

% JCrasher?
% EvoSuite?
% Quviq QuickCheck
\chapter{Work Completed}\label{chapter:work}

%This should discuss what progress has been made on designing, implementing and evaluating the artifact. Care must be taken to ensure that any discussion of technical points are clearly explained, with diagrams being used where appropriate.
%In many cases, the evaluation proper will not yet have begun. However, it is important to demonstrate that sufficient thought has been given to the evaluation.

% Consideration of test generation techniques used
% Trade off => Performance vs accuracy?
% Needed to create test cases that adhered to the pre-conditon

An implementation of the tool has been created with generation of core data types. The tool has also been extended to use integer range analysis.

For a user to use the tool, a user must pass in a WYIL file and specify the test case generation technique to use, number of tests they wish to generate and ranges for generating integers.
% TODO ranges
Ranges are used to bound the integers generated so that ...

\section{Test Case Generation Techniques}

QuickCheck for Whiley currently employs two different techniques to generate test cases: random test-case and exhaustive test case generation.

\subsection{Random Test Case Generation}

A fixed number of tests determined by the user is randomly generated. Previous inputs are not considered which may lead to the same inputs being generated. 
% TODO refer Knuth Algorithm S
Algorithm ... was considered to generate unique inputs but has not been implemented yet.

\subsection{Exhaustive Test Case Generation}
A fixed number of tests determined by the user are exhaustively generated within a bounded range from the minimum bound to the maximum bound. All possible input combinations for a function are tested starting from the combination generated from the minimum bound.

\section{Data types generated}
Different generators are created for the different types defined in Whiley.
Figure XXX illustrates the structure of the generators used.

% TODO add class diagram for generators

% TODO make a list of the types used

% Talk about generators here
% Class diagram of the ranges
% Talk about the different types

The size of arrays are currently limited to a maximum size of three elements due to the performance cost of generating larger arrays.

\section{Integer Range Analysis}

Integer range analysis is ... 
% Refer to paper 

\section{Evaluating the tool}
The tool has been evaluated by executing a variety of tests notably, the test suite for the Whiley Compiler.
%As a result, ... 
% Talk about how the tool is run
% Talk about evaluating the tool using the Whiley valid and invalid tests
% How to evaluate the tool
\section{Future Plan}\label{section:future}

%This should highlight the main components which remain to be done, and provide a proposed time-line in which this will happen. In putting together a time line, students must take into account upcoming examinations, coursework deadlines
%and other disruptions.

A working implementation of QuickCheck for Whiley has been completed halfway through the project. However, there are still several components that need to be completed. The key components are described in the following paragraphs.

\begin{description}
	\item[Optimise function and method calls]
	During execution, functions and methods often call other functions and methods.
	However, calling and executing a function or method is expensive.
	Therefore, it would be beneficial to skip execution by generating the return value of the function or method based on its post-condition.
	As a result of optimising these calls, the performance of the tool will be improved.
	\item[Mutation testing]
	To check if a test can correctly validate a program, a developer would modify a correct implementation of the program and rerun the test.
	Instead of a developer having to do this manually, it would be helpful to automate this process.
	This would involve modifying aspects of a Whiley function or method such as changing a operator in an if statement and then executing a test to see if it is able to detect the error in the mutated function or method.
	The test should pass for a correct implementation of the function or method and fail for the mutated versions.
	\item[Evaluation of the tool] The automatic test case generator will be evaluated against the Whiley Compiler test suite as indicated in Subsection \ref{subsec:toolEval}.
	Mutation testing as discussed in the previous paragraph, will also be used against the Whiley files used for the test suite to generate Whiley files with bugs. 
	This is then evaluated using the different test techniques implemented such as random and exhaustive test case generation to generate tests for the test suite. Various statistics from test case generation will then be compared across the different test techniques including speed of execution, number of successive tests and number of errors detected.
\end{description}

A timeline of the components that need to be completed is shown in Table \ref{table:timeline}.

\begin{table}[H]
	  \centering
\begin{tabular}{ |p{10cm}|p{2cm}|p{2cm}|p{2cm}| }
\hline
\textbf{Output} & \textbf{Estimated Time} & \textbf{Start Date} & \textbf{Complete by}\\
\hline
Optimise function and method calls. & 3 weeks & 11/6/18 & 2/7/18 \\
\hline
Produce slides for presentation of preliminary report. & 1 week & 2/7/18 & 9/7/18\\
\hline
Add mutation testing & 3 weeks & 9/7/18 & 30/7/18 \\
\hline
Produce draft of final report. & 6 weeks & 30/7/18 & 15/9/18\\
\hline
Evaluate the automatic test case generator. & 3 weeks & 15/9/18 & 5/10/18 \\
\hline
Finalise final report. & 2 weeks & 6/10/18 & 21/10/18\\
\hline
Prepare slides for final presentation. & 3 weeks & 22/10/18 & 16/11/18\\
\hline
\end{tabular}

\caption{Timeline}
\label{table:timeline}

\end{table}

% Generation - byte, reference, recursive types? 
\section{Request for Feedback}\label{section:feedback}

%This should highlight any difficulties currently faced, and
%make specific requests for guidance from the examination committee. For example, a student may be unsure how best to evaluate their artifact, and would appreciate
%suggestions for alternative methods.

Feedback about using the Whiley Compiler test suite for evaluating the tool as discussed in Subsection \ref{subsec:toolEval} would be appreciated. 

When evaluating the tool, I would like to know if the following method would be suitable. I could execute the tool multiple times on the same test suite using different test techniques and compare the results based on speed of execution, number of tests passed and number of errors detected. 
Alternative suggestions on how to evaluate the tool would be helpful.


%%%%%%%%%%%%%%%%%%%%%%%%%%%%%%%%%%%%%%%%%%%%%%%%%%%%%%%

\backmatter

%%%%%%%%%%%%%%%%%%%%%%%%%%%%%%%%%%%%%%%%%%%%%%%%%%%%%%%


\bibliographystyle{ieeetr}
%\bibliographystyle{acm}
\bibliography{sample}

\appendix

\chapter{Project Proposal}\label{chapter:proposal}

\includepdf[pages=-]{"../Project Proposal/Project Proposal".pdf}

\end{document}
