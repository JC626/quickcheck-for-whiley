%% $RCSfile: proj_proposal.tex,v $
%% $Revision: 1.4 $
%% $Date: 2017/10/06 02:55:50 $
%% $Author: kevin $

\documentclass[11pt, a4paper, twoside, openright]{report}

\usepackage{float} % lets you have non-floating floats

\usepackage{url} % for typesetting urls
\usepackage{listings}  % for code
%  We don't want figures to float so we define
%
\newfloat{fig}{thp}{lof}[chapter]
\floatname{fig}{Figure}

%% These are standard LaTeX definitions for the document
%%
\title{QuickCheck for Whiley}
\author{Janice Chin}


%% This file can be used for creating a wide range of reports
%%  across various Schools
%%
%% Set up some things, mostly for the front page, for your specific document
%
% Current options are:
% [ecs|msor|sms]          Which school you are in.
%                         (msor option retained for reproducing old data)
% [bschonscomp|mcompsci]  Which degree you are doing
%                          You can also specify any other degree by name
%                          (see below)
% [font|image]            Use a font or an image for the VUW logo
%                          The font option will only work on ECS systems
%
\usepackage[image,ecs]{vuwproject} 

% You should specifiy your supervisor here with
     \supervisors{David Pearce, Lindsay Groves}
% use \supervisors if there are more than one supervisor

% Unless you've used the bschonscomp or mcompsci
%  options above use
   \otherdegree{Bachelor of Engineering with Honour}
% here to specify degree

% Comment this out if you want the date printed.
\date{}

\begin{document}

% Make the page numbering roman, until after the contents, etc.
\frontmatter

%%%%%%%%%%%%%%%%%%%%%%%%%%%%%%%%%%%%%%%%%%%%%%%%%%%%%%%

\begin{abstract}
   This document is a project proposal for the project, QuickCheck for Whiley. 
   The project aims to improve software quality of Whiley programs by  implementing an automated test-case generator for Whiley.
   The solution will be evaluated by checking if it can detect bugs in a small benchmark set. No special resources are required for this project.
\end{abstract}

%%%%%%%%%%%%%%%%%%%%%%%%%%%%%%%%%%%%%%%%%%%%%%%%%%%%%%%

\maketitle

%\tableofcontents

% we want a list of the figures we defined
%\listof{fig}{Figures}

%%%%%%%%%%%%%%%%%%%%%%%%%%%%%%%%%%%%%%%%%%%%%%%%%%%%%%%

\mainmatter

%%%%%%%%%%%%%%%%%%%%%%%%%%%%%%%%%%%%%%%%%%%%%%%%%%%%%%%

\section*{1. Introduction}
Testing is an important process in software development as it helps detect the presence of bugs.
However, writing and running tests can be tedious and costly. Furthermore, it is difficult to write tests for all possible cases and be able to detect all bugs.

An automated test-case generator called QuickCheck was implemented in Haskell by Koen Claessen and John Hughes which alleviates these issues.

This lightweight tool employs two testing ideas.
Firstly, QuickCheck uses property-based testing which uses conditions that will always hold true. Users define the conditions in terms of formal specifications to validate functions under test.

Secondly, QuickCheck generates test cases automatically using random testing. Input values are randomly generated for each test (within a range) using generators that correspond to the input value's type. QuickCheck contains generators for most of Haskell's predefined types and for functions. It requires the developer to specify their own generators for user-defined types. \\

An example of using QuickCheck would be reversing a list.

Firstly, a property is defined where the list xs, should be the same as reversing xs twice.

\begin{lstlisting}
	propReverseTwice xs = reverse (reverse xs) == xs
\end{lstlisting}

QuickCheck is then executed by importing the property and passing into the interpreter:

\begin{lstlisting}
	Main:> quickCheck propReverseTwice
	Ok: passed 100 tests.
\end{lstlisting}

This will create a large number of test cases by using the user-defined property and random input values to check the property holds. QuickCheck reports various test statistics including the number of tests that have passed or failed.

QuickCheck has been re-implemented in other languages including Scala, Java and C++. A commercial version of QuickCheck in Erlang called Quviq QuickCheck is co-founded by one of the original developers, John Hughes and is an extension of the original QuickCheck for Haskell.

This project is about implementing an automated test-case generator for the programming language, Whiley based on the QuickCheck tool. 

%In this section you should include a very brief introduction to the problem and the project.
%
%Your project proposal should cover the following points:
%
%\begin{itemize}
%\item the engineering problem that you are going to solve;
%\item how you plan to solve your problem;
%\item how you intend to evaluate your solution; and
%\item any resource requirements for your project such as software,
%  hardware or other resources that will be needed in the course of the
%  project.
%\end{itemize}
%
%Your proposal should be not more than than 3 pages long.

\section*{2. The Problem}
Whiley is a programming language, developed to verify code and eliminate errors using formal specifications. 
Ideally, programs written in Whiley should contain no errors.
To achieve this goal, Whiley contains a verifying compiler which employs the use of specifications written by a developer to check for common errors such as accessing an index of an array which is outside its boundaries.

Currently, the verifying compiler has limitations when evaluating complex pre- and post-conditions. 
For example, a post-condition could be falsely identified as not holding by the verifying compiler even though the program does meet the post-condition.
Therefore, this project aims to implement an automated test-case generator in Whiley to improve software quality  and increase confidence in unverifiable code.

%In this section you should give a brief description of the problem
%itself. You want to briefly explain the problem, why it is important
%to solve the problem and define your project aims. After reading this
%section, the reader should understand why it is a problem, believe
%that it is important to solve and have a clear idea of the aims of
%your project.
%
%When describing the aims of the project, you should avoid vague,
%unmeasurable words like `analyse', `investigate', `describe', and use
%specific, measurable words like `implement', `demonstrate', `show',
%`prove'.
%
%For example:
%
%\begin{itemize}
%\item[\bf Good] The aim of this project is to implement and evaluate a
%  management system for network switches;
%\end{itemize}
%is much better than:
%\begin{itemize}
%\item[\bf Bad] The aim of this project is to investigate management
%  systems for network switches.
%\end{itemize}
%
%In the second case there is no idea of how much work is involved, and
%you will never know whether you have finished. You and your supervisor
%(and the markers of your project) may have very different ideas about
%what such an `investigation' involves. Of course, it is possible that
%the task you set yourself is not achievable, but if you are clear from
%the outset this is less likely, and will more easily be corrected.

\section*{3. Proposed Solution}

The automated test-case generator needs to read a Whiley program and then generate tests for functions in the program. 
Input values that adhere to the function's preconditions will be used in testing. Strategies for generating input values include random generation and dynamic generation.
The success of each test is determined by using the function's postconditions. 
The test results should show the percentage of tests that passed and the inputs used in the failed tests.

Several components developed for the Whiley language will need to be used for the automated test-case generator. 

After basic test-case generation is completed, more complex types can be generated for testing and/or further enhancements to the tool can be made. \\

Possible enhancements include:
\begin{itemize}
	\item Weight on the frequency of test values
	\item Shrinking failed test cases
	\item Classifying tests
	\item Using dynamic symbolic execution
\end{itemize} 

Regular meetings (weekly or fortnightly) with the primary supervisor, David Pearce will be held for assistance and guidance in the project.

\subsection*{Timeline}
A gantt chart of the project can be found on the project's repository.
\subsubsection*{Trimester 1}
\begin{tabular}{ |p{10cm}|p{2cm}|p{2cm}|p{2cm}| }
	\hline
	\textbf{Output} & \textbf{Estimated Time} &
	\textbf{Start Date} & \textbf{Complete by}\\
	\hline
	Produce bibliography & 2 weeks & 19/3/18
	& 9/4/18 (Week 5) \\
	\hline
	Produce project proposal & 2 weeks & 19/3/18 & 9/4/18 (Week 5) \\
	\hline
	Implement an automated test-case generator for Whiley programs for the primitive types: bool, byte, int, null & 3 weeks & 19/3/18 & 16/4/18 (Week 6) \\
	
	\hline
	Implement automated test-case generation for more complex types: void, array, union and records & 3 weeks & 16/4/18 & 14/5/18 (Week 9) \\
	\hline
	Produce preliminary report & 3 weeks & 14/5/18 & 10/6/18 (Week 12) \\	
	\hline
	Extend the automatic test generator to be able to generate other types or use methods for better test case distribution (weighting, classification) & 3 weeks & 11/6/18 & 2/7/18 (Exam period) \\
	\hline
\end{tabular}

\subsubsection*{Trimester 2}
\begin{tabular}{ |p{10cm}|p{3cm}|p{2cm}|p{2cm}| }
	\hline
	\textbf{Output} & \textbf{Estimated Time} & \textbf{Start Date} & \textbf{Complete by}\\
	\hline
	Produce slides for presentation of preliminary report & 1 week & 2/7/18 & 9/7/18 (Break)\\
	\hline
	Extend the automatic test generator to include other methods of testing (symbolic) or be able to generate other types & 3 weeks & 9/7/18 & 30/7/18 (Week 4)\\
	\hline
	Produce draft of final report & 6 weeks & 30/7/18 & 15/9/18 (Week 7)\\
	\hline
	Complete implementation of automated test generator & 3 weeks & 15/9/18 & 5/10/18 (Week 10) \\
	\hline
	Finalise final report & 2 weeks & 6/10/18 & 21/10/18 (Week 12) \\
	\hline
	Prepare slides for final presentation & 3 weeks & 22/10/18 & 16/11/18 (Exam period)\\
	\hline
\end{tabular}

%In this section you will explain how solve the problem, that is, how
%you intend to carry the project out. At this early stage you need to
%be both clear about what you are going to do and flexible enough to
%adapt to changing circumstances. Making an early plan will not prevent
%you from running into trouble, but it will help you identify possible
%problems early. For example, if you intended to run an experiment in
%HCI, you might realise early on that there would be problems gathering
%sufficient data to get reliable results, and that you should re-design
%your experiment.
%
%Part of the planning process involves producing a timetable for when
%the work is actually going to be done.
%
%Each part of the project should produce some output. For example you
%might plan on spending two weeks on background reading: the output of
%this will be a bibliography, and a possibly a literature survey for
%your report. Indeed, if you take the advice given above about having
%specific, measurable goals, you should describe this part of your
%project as:
%
%\begin{itemize}
%\item[\bf Good] Produce bibliography (est: 2 weeks)
%\end{itemize}
%rather than
%\begin{itemize}
%\item[\bf Bad] Background reading (est: 2 weeks)
%\end{itemize}
%
%Note that the methodology you outline here is dependent upon the type
%of project and engineering area. You must talk to your supervisor
%about this.

\section*{4. Evaluating your Solution}

This project will be evaluated by checking if the developed tool can detect bugs that have been inserted into a small benchmark set.

%In this section you will explain how you will evaluate your solution
%once you have built it. The method of evaluation will be domain
%specific. Your supervisor should provide guidance as to what is an
%appropriate form of evaluation. For example, user testing for a HCI
%project or performance measurement for a Network Engineering project.

\section*{5. Resource Requirements}

No special resources are required. Only the use of the ECS laboratories is required.



%%%%%%%%%%%%%%%%%%%%%%%%%%%%%%%%%%%%%%%%%%%%%%%%%%%%%%%
\backmatter
%%%%%%%%%%%%%%%%%%%%%%%%%%%%%%%%%%%%%%%%%%%%%%%%%%%%%%%

%\bibliographystyle{ieeetr}
%\bibliographystyle{acm}
%\bibliography{sample}
\end{document}
